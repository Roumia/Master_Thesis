\newpage
\begin{savequote}[108mm]
 " Don't mistake activity with achievement."
   \qauthor{John Wooden}
\end{savequote}
\chapter{Conclusions}
\label{chap:conclusions}
\vspace{-2cm}

This research explored the advantages and challenges that are faces by organizations who want to migrate to FLOSS. One reason found for wanting to migrate to FLOSS is for cost savings. The research found that there are many hidden costs and challenges that must be faced in the migration of the software. Understanding scope of the migration process and the needs of the organization are the most important consideration in a successful process. FLOSS can add complexity to an existing system. Compatibility issues are one of the challenges that were found to be a determining factor in the success of the migration process. 

One of the most important considerations to the success of the migration project is providing support for users during and after the implementation process. Because the software itself is not owned by a single corporation, the company no longer has the option of calling the software developer to fix the system or to offer support when it breaks. This is the reason why setting up an in-house support system is an essential part of the success of the migration. Employees must have someone they can call when they have questions or problems. 

It is recommended that government entities have an obligation to exercise oversight of public resources to the greatest extent possible. This means implementing cost saving solutions wherever possible. FLOSS was found to save considerable costs both in the short and long term. There are no more licensing fees, and one does not have to pay to expand the system and add more users. The flexibility of FLOSS is another feature that makes it an excellent choice for organizations and government entities. This opinion is supported by the numerous successes found in the case studies used in this research. 

\section{Evaluation}
The goal of this research was to explore the FLOSS migration in government and private organizations. In meeting that goal, several case studies were found that provided a framework for the development of a migration plan that will be useful in majority of the cases involved in a migration from CSS to FLOSS. This research was successful in exploring the advantages of migrating to an FLOSS system. It also explored the challenges and developed a plan that will result in attention to the major areas that were found during the case study review.

The case studies identified supported the premise of the study and the advantages of FLOSS for government entities. It sufficiently addressed the intended answers that it hoped to find. The research demonstrated that although cost is a fact in the decision to migrate from CSS to FLOSS, this not the only factor that is considered. Outpacing cost as a reason for migrating to FLOSS, the organization has the freedom to design a system that specifically meets their needs. When one purchases CSS, they are getting a prepackaged  software, it will often have things that they do not need, yet be lacking in the features that they do need. FLOSS gives companies the freedom that they need to design a system that is the perfect fit.          
%----------------------------------------------------------------------------------------------
\section{Lessons learned}
\label{sec:lessons}
Several lessons were learned throughout the course of this research study. 
\begin{itemize}[itemsep=0ex]
\item  The first lesson is that is not only possible, but necessary for government entities to take advantage of the savings and flexibility that FLOSS has to offer.
\item It is necessary to convince people to be ready and open for change.
\item One of the key goals of the migration process is to make the transition from proprietary software to FLOSS as seamless as possible. The goal is to make the change without sacrificing business continuity. Compatibility issues can bring the business to a halt during the migration process. This can cost the company untold losses in terms of monetary losses.
\item The critical phase of the migration plan is to determine the needs of the organization and to set measurable goals for completion of the project.
\item Planning is the most important phase of the migration process.

\end{itemize}

In a personal aspect through conducting this study, gaining the knowledge and skills necessary to successfully implement the migration from CSS to FLOSS. having the ability to plan and execute a migration plan. Familiar with the challenges and obstacles that could be faced in the implementation and execution of the project. 
This research provided me with the opportunity to expand my knowledge and skills in the area of software migration and FLOSS for small, medium, and large government entities.
\subsection{Knowledge and Skills Acquired }
The following explains knowledge and skills acquired in the M.Sc. studies that helped me on
this work:
	\begin{enumerate}[itemsep=0ex]
	
\item \textbf{Introduction to Libre Soft.}

 This course explored the fundamentals of Libre Software and its history and evolution. This research project added to the knowledge learned in class through allowing me to explore how it is used in real world applications.

\item \textbf{Legal Aspects of Libre Software.} 

This course explored the legal aspects of Libre Software including copyright, licensing, patents, lawsuit, and other legal issues. This course helped me to understand the types of issues involved when choosing an FLOSS licensing. 

\item \textbf{Economic Aspects of Libre Software.}

 This course explored the economic motivations behind FLOSS. This course played an essential role in helping to build the case for encouraging organization to begin using FLOSS. Although the economics are an important part of the motivation to switch, it was found that other factors, such as the ability to design a customer system to meet their needs. This turned out to be more important than the economic reasons for the switch. 
 
\item  \textbf{Case Studies.}

Thanks to this course I met \textit{Eduardo Romero}, Computer technician at City of Zaragoza. He inspired and encourage me to do this work, his speech was helpful for me to do this work. and I am looking forward to move his experience back home. 
                 
   \item \textbf{Developers and Motivation of LibreSoft.}
                 
 This course taught me the motivations of libre software developers, roles and organization in libre software projects and the leadership in libre software projects. Much of the information obtained in this course was be applied when studying the migration motivation and plans, and how to lead and organize the migration.
 
 \item \textbf{System Integration. }
 
 This course provided information on system platforms, virtualization, and network administration of FLOSS. This course was valuable in helping me to identify and foresee challenges that could occur during the migration.
 
\item \textbf{Project Management }

This course provided me with the knowledge to break the task down into manageable phases to be carried out. It was inspiration for the organization of the migration process. for instance documentation and communication channels can be good way for reaching new contributors and stability of the projects. 

\item \textbf{Project Evaluation of Libre Soft, Development Tools of Libre Soft.}

These courses taught me to take into account every aspect of the software beyond the code or how it works. I found out how to use the control version system that I used to keep track of each change made to this document. This project made me look at the impact of the software on the organization as a whole. It made me look at the bigger picture beyond the code.  I also learned the importance of FLOSS project evaluation, which I was able to apply in the development of the migration plan. 

\item\textbf{ Advanced Development. }

This class taught how to make Android software. This course  provided me with background on some of the compatibility problems that may be encountered when transferring between platforms.          

                                                                                               
	\end{enumerate}
\section{Future work}
\label{sec:future}
In the future, this work will continue through seeking out more case studies so that the experience can be added to the overall knowledge on the topic. The more case studies that are read, the broader the knowledge base will be and the easier it will be to apply that knowledge to the more migration projects. Case studies provide examples and insight as to how to overcome challenges that may arise. 

In addition to finding more case studies and increasing the knowledge base, the researcher will provide a specific plan for a migration. The researcher will then attempt a migration to FLOSS. This future work will provide practical field experience and the ability to see what types of issues arise that were not discovered as part of this research study.  Finally, future research will entail a comparison of different types of FLOSS that provide the best migration experience. This will help organizations to choose the right FLOSS for their migration.                                                                                                                                                                                                                                                                                                                                                                                                                                                                                                                                                                                                                       



